% このファイルは日本語用です。
% 次の行は変更しないでください。
\documentclass[ja]{2020a}
%%%%%%%%%%%%%%%%%%%%%%%%%%%%%%%%%%%%%%%%%%%%%%%%%%%%%%%%%%%%%%%%
% 講演者についての情報
\PresenterInfo
%%%%%%%%%%%%%%%%%%%%%%%%%%%%%%%%
% 講演数(半角数字)
{1}
%%%%%%%%%%%%%%%%%%%%%%%%%%%%%%%%
% 氏名
{}
%%%%%%%%%%%%%%%%%%%%%%%%%%%%%%%%
% 氏(ひらがな, 氏名が英字の場合はalphabet)
{}
%%%%%%%%%%%%%%%%%%%%%%%%%%%%%%%%
% 名(ひらがな, 氏名が英字の場合はalphabet)
{}
%%%%%%%%%%%%%%%%%%%%%%%%%%%%%%%%
% 所属機関(機関名(◯◯大学、◯◯研究所、など)のみ)
{}
%%%%%%%%%%%%%%%%%%%%%%%%%%%%%%%%
% 会員種別(半角英小文字)
%   a=正会員(一般)
%   b=正会員(学生)
%   c=準会員(一般)
%   d=準会員(学生)
%   e=非会員(一般)〔企画セッションのみ〕
%   f=非会員(学生)〔企画セッションのみ〕
{}
%%%%%%%%%%%%%%%%%%%%%%%%%%%%%%%%
% 会員番号(半角数字4桁)
%   入会申請中の場合、受付番号(半角英大文字+半角数字5桁)
{}
%%%%%%%%%%%%%%%%%%%%%%%%%%%%%%%%
% メールアドレス(半角)
{}
%%%%%%%%%%%%%%%%%%%%%%%%%%%%%%%%%%%%%%%%%%%%%%%%%%%%%%%%%%%%%%%%
% 講演についての情報
\PaperInfo
%%%%%%%%%%%%%%%%%%%%%%%%%%%%%%%%
% 記者発表(半角英小文字)
%   申請する場合のみ「y」を記入
{}
%%%%%%%%%%%%%%%%%%%%%%%%%%%%%%%%
% 講演分野(半角)
%   [通常セッション]
%     M=太陽
%     N=恒星・恒星進化
%     P1=星・惑星形成(星形成)
%     P2=星・惑星形成(原始惑星系円盤)
%     P3=星・惑星形成(惑星系)
%     Q=星間現象
%     R=銀河
%     S=活動銀河核
%     T=銀河団
%     U=宇宙論
%     V1=観測機器(電波)
%     V2=観測機器(光赤外・重力波・その他)
%     V3=観測機器(X線・γ線)
%     W=コンパクト天体
%     X=銀河形成
%     Y=天文教育・広報普及・その他
%   [企画セッション]
%     Z1=ALMAとすばるのシナジーによる銀河研究
%     Z2=SPICAが切り拓くサイエンス
%     Z3=天文学史の最新研究動向
%     Z4=突発現象天文学と大学教育における大学望遠鏡のシナジー
{}
%%%%%%%%%%%%%%%%%%%%%%%%%%%%%%%%
% 講演形式(半角英小文字)
%   a=口頭講演
%   b=ポスター講演(口頭有)
%   c=ポスター講演(口頭無)
{}
%%%%%%%%%%%%%%%%%%%%%%%%%%%%%%%%
% キーワード(5つまで)
%   分野Y以外は PASJ keyword list から選択
{}
{}
{}
{}
{}
%%%%%%%%%%%%%%%%%%%%%%%%%%%%%%%%
% 題名
{}
%%%%%%%%%%%%%%%%%%%%%%%%%%%%%%%%
% 氏名及び所属(複数の場合は「, 」で区切)
{}
%%%%%%%%%%%%%%%%%%%%%%%%%%%%%%%%%%%%%%%%%%%%%%%%%%%%%%%%%%%%%%%%
\begin{document}
%%%%%%%%%%%%%%%%%%%%%%%%%%%%%%%%%%%%%%%%%%%%%%%%%%%%%%%%%%%%%%%%
% 本文開始
%%%%%%%%%%%%%%%%%%%%%%%%%%%%%%%%%%%%%%%%%%%%%%%%%%%%%%%%%%%%%%%%

% 日本天文学会の年会において講演を行うためには、指定された期日ま
% でに予稿の提出と「講演登録費」の支払を済ませておく必要がある。
% ここでは、講演申し込みにまつわる注意点を記す。

% まず最初に、予稿は非専門家にもわかりやすく書かなければならない
% ことを挙げたい。研究の背景はもちろんのこと、研究手法や解析方法、
% 得られた結果やそれに基づく議論などをやさしく記すべきである。そ
% うすると、おのずとある程度の分量になるだろう。極端に短い予稿は、
% 受理されない場合もあることを注意しておく。

% つぎに、講演に伴う費用について説明したい。予稿が受け付けられる
% と、講演者は、定められた期限までに「講演登録費」を支払う義務が
% 生じる。「講演登録費」は文字通り講演の登録や予稿集印刷等に関わ
% る費用であり、講演が登録された時点で発生し消滅することはない。
% したがって、たとえ実際の講演者が交代したり講演をキャンセルした
% りしたとしても、「講演登録費」の返金は行われない。

% 講演を申し込んだ後、何らかの事情によって講演の実施が困難になっ
% た場合は、代理講演者を立てるか、講演をキャンセルすることになる。
% いずれの場合も、早めに年会実行委員会に連絡することが望ましい。

% 予稿の著作権についても触れておく。予稿として投稿されたものは、
% 著者から日本天文学会に著作権が委譲されたものとみなす。したがっ
% て、公開された予稿はすべて、日本天文学会が著作権をもっている。

% 最後に、「お知らせ」をもう一度よく読んでいただくことをお願いし
% ておきたい。しかしそれでも何か質問等がある場合は、年会実行委員
% 会まで遠慮なくお問い合わせいただきたい。

%%%%%%%%%%%%%%%%%%%%%%%%%%%%%%%%%%%%%%%%%%%%%%%%%%%%%%%%%%%%%%%%
% 本文終了
%%%%%%%%%%%%%%%%%%%%%%%%%%%%%%%%%%%%%%%%%%%%%%%%%%%%%%%%%%%%%%%%
\end{document}
