% このファイルは日本語用です。
% 次の行は変更しないでください。
\documentclass[ja]{2020a}
%%%%%%%%%%%%%%%%%%%%%%%%%%%%%%%%%%%%%%%%%%%%%%%%%%%%%%%%%%%%%%%%
% 講演者についての情報
\PresenterInfo
%%%%%%%%%%%%%%%%%%%%%%%%%%%%%%%%
% 講演数(半角数字)
{1}
%%%%%%%%%%%%%%%%%%%%%%%%%%%%%%%%
% 氏名
{谷口暁星}
%%%%%%%%%%%%%%%%%%%%%%%%%%%%%%%%
% 氏(ひらがな, 氏名が英字の場合はalphabet)
{たにぐち}
%%%%%%%%%%%%%%%%%%%%%%%%%%%%%%%%
% 名(ひらがな, 氏名が英字の場合はalphabet)
{あきお}
%%%%%%%%%%%%%%%%%%%%%%%%%%%%%%%%
% 所属機関(機関名(◯◯大学、◯◯研究所、など)のみ)
{名古屋大学}
%%%%%%%%%%%%%%%%%%%%%%%%%%%%%%%%
% 会員種別(半角英小文字)
%   a=正会員(一般)
%   b=正会員(学生)
%   c=準会員(一般)
%   d=準会員(学生)
%   e=非会員(一般)〔企画セッションのみ〕
%   f=非会員(学生)〔企画セッションのみ〕
{a}
%%%%%%%%%%%%%%%%%%%%%%%%%%%%%%%%
% 会員番号(半角数字4桁)
%   入会申請中の場合、受付番号(半角英大文字+半角数字5桁)
{5892}
%%%%%%%%%%%%%%%%%%%%%%%%%%%%%%%%
% メールアドレス(半角)
{taniguchi@a.phys.nagoya-u.ac.jp}
%%%%%%%%%%%%%%%%%%%%%%%%%%%%%%%%%%%%%%%%%%%%%%%%%%%%%%%%%%%%%%%%
% 講演についての情報
\PaperInfo
%%%%%%%%%%%%%%%%%%%%%%%%%%%%%%%%
% 記者発表(半角英小文字)
%   申請する場合のみ「y」を記入
{}
%%%%%%%%%%%%%%%%%%%%%%%%%%%%%%%%
% 講演分野(半角)
%   [通常セッション]
%     M=太陽
%     N=恒星・恒星進化
%     P1=星・惑星形成(星形成)
%     P2=星・惑星形成(原始惑星系円盤)
%     P3=星・惑星形成(惑星系)
%     Q=星間現象
%     R=銀河
%     S=活動銀河核
%     T=銀河団
%     U=宇宙論
%     V1=観測機器(電波)
%     V2=観測機器(光赤外・重力波・その他)
%     V3=観測機器(X線・γ線)
%     W=コンパクト天体
%     X=銀河形成
%     Y=天文教育・広報普及・その他
%   [企画セッション]
%     Z1=ALMAとすばるのシナジーによる銀河研究
%     Z2=SPICAが切り拓くサイエンス
%     Z3=天文学史の最新研究動向
%     Z4=突発現象天文学と大学教育における大学望遠鏡のシナジー
{V1}
%%%%%%%%%%%%%%%%%%%%%%%%%%%%%%%%
% 講演形式(半角英小文字)
%   a=口頭講演
%   b=ポスター講演(口頭有)
%   c=ポスター講演(口頭無)
{b}
%%%%%%%%%%%%%%%%%%%%%%%%%%%%%%%%
% キーワード(5つまで)
%   分野Y以外は PASJ keyword list から選択
{methods: data analysis}
{methods: observational}
{methods: statistical}
{}
{}
%%%%%%%%%%%%%%%%%%%%%%%%%%%%%%%%
% 題名
{単一鏡観測装置開発のための共通データ解析ソフトウェアの開発}
%%%%%%%%%%%%%%%%%%%%%%%%%%%%%%%%
% 氏名及び所属(複数の場合は「, 」で区切)
{谷口暁星, 立原研悟(名古屋大学), 竹腰達哉, 石田剛, 吉村勇紀(東京大学), 新田冬夢, Pranshu Mandal, 村山洋佑(筑波大学), 大島泰, 永井誠, 川邊良平(国立天文台)}
%%%%%%%%%%%%%%%%%%%%%%%%%%%%%%%%%%%%%%%%%%%%%%%%%%%%%%%%%%%%%%%%
\begin{document}
%%%%%%%%%%%%%%%%%%%%%%%%%%%%%%%%%%%%%%%%%%%%%%%%%%%%%%%%%%%%%%%%
% 本文開始
%%%%%%%%%%%%%%%%%%%%%%%%%%%%%%%%%%%%%%%%%%%%%%%%%%%%%%%%%%%%%%%%

地上の(サブ)ミリ波単一鏡における時系列観測と解析ソフトウェア開発は、装置の応答関数較正と大気雑音除去を達成するために必要不可欠である。
広帯域化・広視野化に対する需要が高まる中、大学主導の新規装置開発や観測手法開発が現在精力的に行われている(e.g., DESHIMA; Endo et al. 2019a/b, FMLO; Taniguchi et al. 2019, MKIDカメラ; 永井他2019年春季年会, 2--mm受信機; 酒井他2019年春季年会., TESカメラ; 大島他2017年春季年会, NASCO受信機; 山本他2020年春季年会)。
これらの観測データは時刻に対して周波数・偏波・空間情報の複数チャンネルが並ぶという時系列構造を持つため、相関雑音除去や周波数解析などの処理を共有できる可能性がある。
一方、解析ソフトウェアは各々のプロジェクトで独自開発されているのが現状である。

本講演では、CASAなどによる共同利用単一鏡のデータ解析と相補的な、新規装置開発の即応性や柔軟性に応えるための共通解析ソフトウェア\texttt{sdarray}(https://github.com/sdarray)を紹介する。
\texttt{sdarray}は、大規模データの並列計算やメタデータを扱うことのできる\texttt{xarray}(Hoyer et al. 2017)を採用した\texttt{Python}パッケージとして、現在初期段階の開発を行っている。
時系列構造は、時間軸と周波数・偏波・空間情報を平坦化したチャンネル軸の2次元配列で表現される。
これにアンテナ座標値やヘッダなどの0--1次元配列のメタデータを付加することで、観測データを1つの\texttt{Python}オブジェクトとして扱うことができ、\texttt{NumPy}と同様の記法で配列計算にも対応する。
本講演では、上記のプロジェクトの観測データに加え、FITSやCASA Measurement Sets (MS)など一般的なデータ形式を考慮した\texttt{sdarray}の初期仕様と、\texttt{sdarray}に実装する予定の共通処理も紹介する。

%%%%%%%%%%%%%%%%%%%%%%%%%%%%%%%%%%%%%%%%%%%%%%%%%%%%%%%%%%%%%%%%
% 本文終了
%%%%%%%%%%%%%%%%%%%%%%%%%%%%%%%%%%%%%%%%%%%%%%%%%%%%%%%%%%%%%%%%
\end{document}
